\documentclass[12pt]{article}

\usepackage{tikz}
\usetikzlibrary{calc}

\usepackage{graphicx}
\graphicspath{{Images/}}

\begin{document}
\begin{tikzpicture}
[remember picture, overlay]      \draw[line width = 3pt]  ($(current page.north west)+ (0.2in, -0.3in)$)     rectangle    ($(current page.south east)+ (-0.2in, 0.3in)$);
\end{tikzpicture}



\begin{center}
\huge {\textbf{Term Project}}\\
\Large {\textbf{On}}\\
\Large {\textit{Artificial Intelligence in Healthcare}}\\
\hspace{2cm}

\large {\textsc{Submitted By}}\\
\large {\texttt{Harshita Upendra Sakhare}}\\
\large {\texttt{Roll.No :- 21111049}}\\
\hspace{2cm}

\large {\textbf{1st Semester Biomedical Engineering}}

\hspace{2cm}


\Large {\textbf{National Institute of Technology Raipur}}
\end{center}


\begin{figure}[h]
\centering
\includegraphics[scale=0.15]{NITRR.jpg}
\end{figure}

\begin{center}
\large {\textsc{Under the Supervision of :-}}\\
\large {\texttt{Prof. Saurabh Gupta}}\\
\large {\texttt{Biomedical Department}}\\
\large {\texttt{National Institute Of Technology Raipur.}}
\end{center}

\clearpage
\tableofcontents
\clearpage

\section{Acknowledgement}

\hspace{1cm}

I thank Prof. Saurabh Gupta, Biomedical Department,  for his excellent supervision of my quarterly project on "Artificial Intelligence in Healthcare". Sir, I am very grateful for your guidance and support. 
 
I would like to use this time to express my gratitude to my friends and family. 
Without them, this assignment could not have been completed in such a short time.


\clearpage


\section{Abstract}
\hspace{1cm}
Artificial intelligence (AI) is the ability of a computer or computer-controlled robot  to perform tasks that normally require human intelligence (the ability to acquire and apply knowledge and skills) and judgment (the ability to make judgments). AI is a broad idea that has applications in a variety of fields, including transportation, manufacturing, advertising, finance, and healthcare. In this  project, I will provide a brief overview of artificial intelligence in healthcare and how it could benefit a doctor.

\section{Keywords}

Artificial Intelligence, Artificial neural networks, Machine learning, Robots, future of healthcare

\section{Introduction}
\hspace{1cm}
Alan Mathison Turing, an English mathematician, logician, and cryptographer, was a key figure in the development of modern computers and artificial intelligence (AI). The "Turing test" is a method used in artificial intelligence (AI) to test whether  a computer is capable of thinking like a human. AI attracted a great deal of interest in the 1980s and 1990s Artificial intelligence approaches such as fuzzy expert systems, Bayesian networks, artificial neural networks and hybrid intelligent systems have been used in a variety of clinical situations.
 In medicine, artificial intelligence is divided into two types: virtual and physical. The virtual part includes everything from electronic patient record systems to neural network-based treatment decision aids. The physical aspect refers to the robots that help carry out tasks. Surgical interventions, smart prosthetics for people with disabilities and elderly care are just a few examples.

\clearpage

\section{What is Artificial Intelligence(AI) in\\ Healthcare?}
\hspace{1cm}
The use of machine learning (ML) algorithms and other cognitive technologies in the medical context is referred to as artificial intelligence (AI). Basically, AI refers to the ability of computers and other technologies to mimic human cognition, including the ability to learn, think, and make decisions. In healthcare, artificial intelligence (AI) refers to the use of machines to analyze and act on medical data, usually with the goal of predicting a specific outcome. 
 

\begin{figure}[h]
\centering
\includegraphics[scale=0.3]{AI in HC.jpg}
\caption{Artificial Intelligence in Healthcare}
\end{figure}

The use of machine learning (ML) and other cognitive sciences for medical diagnosis is an important application of AI  in healthcare. AI can help doctors and healthcare professionals make more accurate diagnoses and treatment plans by using patient data and other information. By analyzing data, AI can also help make healthcare more predictive and proactive. Big data to give patients better  recommendations for screening




\section{How does AI Helps in Medical Sector ?}
\hspace{1cm}
When many of us hear the term "artificial intelligence" (AI), we imagine robots doing our jobs, rendering people obsolete. And, since AI-driven computers are programmed to make decisions with little human action, some wonder if machines will soon make the difficult decisions we now assign the responsibility for making decisions to our doctors.

Rather than robotics, AI in health care mainly refers to doctors and hospitals accessing vast data sets of potentially life-saving information. This includes treatment methods and their outcomes, survival rates, and speed of care gathered across millions of patients, geographical locations, and innumerable and sometimes interconnected health conditions. New computing power can detect and analysis large and small trends from the data and even make predictions through machine learning that's designed to identify potential health outcomes.

Machine learning uses statistical techniques to give computer systems the ability to "learn" with incoming data and to identify patterns and make decisions with minimal human direction. Armed with such targeted analytics, doctors may be better able to assess risk, make correct diagnoses, and offer patients more effective treatments

\section{Difference Between Artificial Intelligence(AI) and Machine Learning(ML)}
\hspace{1cm}
Artificial intelligence (AI) is a technology that enables a computer to mimic human behaviour. System learning is a subset of AI that allows a machine to learn from previous data without having to explicitly design it. The goal of AI is to create an intelligent computer system that can solve complicated problems in the same way humans can. In AI, we create intelligent computers that can perform any task in the same way as a human. In machine learning, we use data to train machines  to perform a task. and deliver reliable results. The two most important subfields of AI are machine learning and deep learning. The most important subset of machine learning is deep learning. 

AI offers wide range of applications. Machine learning has limited utility. Yourself complete a variety of difficult tasks Machine learning aims to build machines that can only perform the tasks they were programmed to do. The AI system focuses on increasing the probability of success. Machine learning is all about pattern recognition and accuracy. Siri, catboat customer service, expert systems, online games, intelligent humanoid robots and other AI applications are among the most common. .Online recommendation system, Google search algorithms, automatic tagging suggestions from Facebook friends and other machine learning applications are the most common.

 

\section{Applications of Artificial Intelligence in\\Healthcare}

1. Medical Records and alternative knowledge Management\\
2. operating in Repetitive Positions\\
3. Treatment coming up with\\
4. Consultation via the web\\
5. Nurses who work remotely\\
6. Medication Administration\\
7. Drug Development\\
8. Medical exactitude\\
9. Keep track of your health.\\
10. Examine the attention System\\



\section{Conclusion}
AI has the potential to transform medicine in previously unimagined ways, but many of its practical applications are still in the early stages of development and need to be thoroughly studied and developed. Medical practitioners must also comprehend and acclimate to these advancements in order to provide better healthcare to the general public.


\section{Reference}

\begin{itemize}
\item Wikipedia
\item ncbi.nim.nih.gov
\item analyticsinsight.net
\item medicalfuturist.com
\end{itemize}

























\end{document}